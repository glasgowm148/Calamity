\chapter{Conclusion And Future Work}

Earlier many people have tried various adhoc approaches on micro-blogs to detect, track and analyse events. In our work we have explored a different approach by using topic models on tweets to cluster them according to the topics they belong. This clustering of tweets helps in organizing the micro-blog information into high-level abstraction-based clusters. But, a high-level topic alone does not represent a specific event instance. Instead, a topic may have numerous instances of different events which come under the purview of the broad topic. To this end, we have proposed a hierarchical approach by temporally segmenting the tweets in a topic and then finding the named-entities present in the tweets sub-cluster. After merging related entities, each group of tweets represents an event instance. The advantage of the hierarchical approach is that further hierarchies based on additional features can be added to the pipeline to add more specificity to the result. For example, since location is an important attribute associated with an event, spatial segmentation could be added to create a 3-level hierarchical framework. Our tests on a Twitter dataset having 1.6 million Tweets shows promising results, with many event instances extracted from the corpus.

After event detection we track events over time to study their evolution. We have modeled the problem as an instance of maximum bipartite matching. We have achieved this by taking intersection of keywords words describing an event with events on another time-frame. This simple event tracking technique used by us looks very promising for tracking events as we have used weighted edges to show relationship between events. These weighted edges also give a good idea of the degree of correlation between events.

There is a lot of potential in carrying our analysis further. One direction of work could focus specifically on development of variants of Topic Modeling techniques specific to micro-blogs. Our current framework clusters tweets at the bottom level using direct entity presence. While this approach works reasonably well, it is admittedly crude and simplistic. More involved approaches such as supervised/semi-supervised clustering techniques such as Labelled LDA \cite{ramage2009labeled} could be employed at this level to extract even non-popular event instances which might otherwise get subdued by trending ones. Finally, more involved time-series models such as ARIMA could be employed for event tracking.